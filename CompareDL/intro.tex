\section{Introduction}
As networking technology gets deeply integrated into our lives, protecting modern networked systems against cyber-attacks is no longer optional.
Network intrusion detection systems (NIDS) are essential security solutions for today's networked systems supporting military applications,
social communications, cloud services, and other critical infrastructures.
A NIDS automatically monitors traffic in a network to detect malicious activities and policy violations.
The majority of NIDSes today adopt signature-based detection techniques,
which can only identify known attacks via matching pre-installed signatures to observed network activities. 
The signature databases have to be frequently updated to include new types of attacks.
Those limitations have motivated researchers to investigate anomaly detection based approaches~\cite{STL-NIDS, LOF, RankingOutliner, NB-Tree, RampLossKSVCR, GAA-ADS}. 

Anomaly detection approaches use data mining or machine learning techniques to mathematically model the trustworthy network activities based on a set of training data,
and detect deviations from the model in observed data. A key advantage is the ability to detect unknown or novel malicious activities.
An on-line model further frees network administrators from identifying new patterns or even new types of the abnormal behaviors in a dynamic network environment.
However, if the constructed model is not sufficiently generalized for normal or abnormal traffic,
anomaly-based approaches would suffer from high false positive, i.e., incorrectly treat unknown normal traffic as malicious.

Deep learning has gained a dramatic increase in popularity in the last couple of years,
and has offered advanced solutions in the areas of image and speech recognition~\cite{AlexNet, SpeechDNN},
natural language processing~\cite{Word2Vec}, Go playing~\cite{AlphaGo}, and many other domains~\cite{DeepLearning}. 
This motivates us to study the feasibility to enhance the anomaly detection based NIDS with the state-of-art deep neural networks trained by innovative algorithms.
We also study the unsupervised generative deep learning models for intrusion detection,
because those models can extract useful and hierarchical features from the vast amount of unlabeled traffics. 

In this paper, we introduce a bundle of deep learning models for the network intrusion detection task,
including multilayer perceptron, restricted Boltzmann machine, sparse autoencoder, and wide \& deep learning.
We also develop and open-source our TensorFlow-based testing and evaluation platform, NetLearner~\cite{NetLearner}, to the research community.
NetLearner includes the implementations of the studied deep learning models as well as the training procedures,
which facilitates the reproduction and further extension of this work.
Finally, we conduct a quantitative and comparative study of those deep learning models using two network intrusion detection datasets
(i.e., NSL-KDD~\cite{NSL-KDD} and UNSW-NB15~\cite{UNSW}) and measure the accuracy, precision, and recall.
Our experimental results show that for the NSL-KDD task, sparse autoencoder achieves an equivalently good performance to the existing machine learning solutions;
and for the UNSW-NB15 task, the deep neural network models with greater generalization capability achieve better accuracy than support vector machine (SVM) models.

\if 0
Security has become a crucial issue for all modern computing and communication systems.
Network intrusion detection system (NIDS) is the essential protection technology for networked systems.
It automatically monitors a networked system and detect malicious activities or policy violations.
Most current NIDSs adopts that signature based approach, e.g., SNORT~\cite{Snort}, in which
pre-installed rules for specific attacks are used as signature to match known attacks.
The fatal drawback of signature based NIDS is that
it is only effective for previously detected attacks that have an identifiable signature.
As a result, signature database needs to be manually updated whenever a new type of attack
is discovered, with significant effort, by the network administrator.

Therefore, researchers have investigated anomaly detection based approach~\cite{STL-NIDS, LOF, RankingOutliner, NB-Tree, RampLossKSVCR, GAA-ADS},
which overcomes these limitations by adopting data mining or machine learning technique to
model the trustworthy network activities mathematically.
Network traffics that significantly deviate from the built model are treated as malicious.
The advantage of anomaly detection based approach is its ability to detect previously unknown or novel malicious activities.
An on-line model further frees the network administrators from identifying new patterns,
or even new types of the abnormal behaviors in the constantly updating network environment.
However, if the built model for normal or abnormal traffics are not generalized enough,
anomaly based approach will treat unforeseen normal traffic as malicious,
suffering from high false positive.

Deep learning recently catch almost everyone's eye because it achieves significant improvement in image and speech recognition~\cite{AlexNet, SpeechDNN}, natural language processing\cite{Word2Vec}, Go playing~\cite{AlphaGo} and many other domains~\cite{DeepLearning}. 
In this project, we follow the anomaly detection based idea, and tries to enhance it with the
state-of-art machine learning technology, e.g., various deep neural networks trained by innovative algorithms.
We expect these innovative deep learning algorithms and training techniques would directly
help neural networks to achieve better performance in the network intrusion detection problem.
In addition, the unsupervised generative deep learning models are also promising solutions to malicious traffic detection
because they are able to extract useful and hierarchical features from vast amount of unlabeled traffics.
Specifically, we have made the following contributions:
\begin{itemize}
    \item Firstly, we introduce a bundle of deep learning models for the network intrusion detection task,
    including multilayer perceptron, restricted Boltzmann machine, sparse autoencoder, and wide \& deep learning.
    \item We develop and open-source the NetLearner~\cite{NetLearner} to the research community,
    including the implementations of introduced models as well as training procedures,
    both facilitating the reproduction and further extension of this work. 
    \item We conduct a quantitatively comparative study of considered deep learning models with
    two network intrusion detection datasets~\cite{NSL-KDD, UNSW}, measuring accuracy, precision and recall.
\end{itemize}
Preliminary experimental results show that for the NSL-KDD task, 
sparse autoencoder achieves similar performance to the existing machine learning solutions;
deep neural network models with greater generalization capability
achieve better accuracy than SVM for the UNSW-NB15 dataset.
\fi
