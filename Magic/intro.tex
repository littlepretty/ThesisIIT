
\Subsection{Deep Learning Based Malware Classifier}
\label{DL:SubSec:MC}
% The harm of malicious software in current days.
While many anti-virus vendors and computer security researchers have fought hard against malicious software for many years,
they remain one of the biggest digital threats in the cyber world.
Motivated by the high return on investment ratio,
the underground malware industry has been consistently enlarging the sheer volume of the threats on the Internet every year.
According to a report by AV-TEST~\cite{AvTest}, the number of total malware by 2018 is estimated at over 800 million,
which has increased 28 times over the past 10 years.
Even worse, recently viciously keen cyber criminals made a lot of efforts to diversify their avenues of attacks~\cite{SymantecReport}.
Inspired by the astonishing rise in crypto currency values,
47 new coin mining malware families have emerged since early 2017 and was ascribed to the 956\% soar of the number of crypto mining attacks in the past year~\cite{CryptoMiningAttacks}.
% Similarly, corresponding to the fact that more and more network traffics are coming from mobile devices instead of desktops,
% mobile malware continuous to surge year over year\cite{SymantecReport}.
Traditional signature-based detection approaches have failed to thwart the ever-evolving malware threats.
Therefore efficient, robust and scalable anti-malware solutions are indispensable for protecting the trustworthiness of the modern cyber world.

% Weakness of existing approaches
A number of recent research efforts~\cite{NgramMalwareDetect, QgramMalwareDetect, McBoost, GraphMalwareDetect, YanDataset, AutoEncoderFeatureLearn, AutoEncoderMicrosoft, FunctionCallGraph, YuxinMalwareDnn,NovelFeatureFusion,StaticFeatures,PolySeqCls} have shown that machine learning (ML) offers a promising approach to thwart the voluminous malware attacks.
These efforts have been inspired by the great success of ML in improving the accuracy of image classification, language translation, and many other applications.
There is however a dilemma when we investigate an effective ML-based malware classification system.
On one hand, if we focus too much on finding discriminative malware features to achieve high classification accuracy, the features constructed as such may lose their generality when a different operational environment is encountered.
For instance, although features extracted from the PE headers of malware files have been shown useful in classifying PE malware variants belonging to different families~\cite{yan2013exploring},
a malware classification system trained with these features cannot be used to detect fileless malware that only exists in the memory.
On the other hand, if the features extracted from malware programs are too generic, such as the frequencies of n-gram byte sequences~\cite{NgramMalwareDetect},
it is difficult to train a malware classifier with high accuracy from them due to their lack of discriminative power~\cite{yan2013exploring}.
%These efforts are inspired by the status quo that ML and DL algorithms have proven high performance in
%image classification, language translation, and many other research fields\cite{DeepLearning}.
% once informative and discriminating features are learned and extracted.
%Still, feature selection makes existing ML- and DL-based detection approaches suffer one or more of the following weaknesses.
%First, people have to spend significant amount of time on, instead of designing smart algorithms, searching non-intuitive and indirect features.
%Secondly, since features are manually picked from a specific dataset, they may not be commonly available in all situations.
%For example, as the most informative part of a binary, PE-header is not always accessible due to varying reasons\cite{YanDataset}.
%Besides, certain golden features are not robust in the face of polymorphic and metamorphic malware.

To strike a balance between generality and performance, we aim to build a malware classification system from malware programs represented as their control flow graphs (CFGs), a data structure commonly used to characterize the control flow of any computer program.
The generality of CFGs for malware classification can be attributed to two factors:
(1) the CFG can be extracted from different formats of malware code, such as binary executable files, exploit code discovered in network traffic~\cite{zhang2007analyzing}, emulated malware~\cite{sharif2009automatic},
and attack code chained together from gadgets in return-oriented programming attacks~\cite{shacham2007geometry},
and (2) the CFG can be used to derive a variety of static analysis features widely used in existing works on ML-based malware classification,
such as n-grams~\cite{NgramMalwareDetect}, q-grams~\cite{QgramMalwareDetect}, opcodes~\cite{bilar2007opcodes} and structural information~\cite{kong2013discriminant}.
Therefore, a malware classification system trained from CFG-represented malware programs can find applications in various operational environments.

% How CFG and deep learning can help.
%We think the key to overcome these weaknesses is to fully utilize a special kind of representation of executable binary: Control Flow Graph (CFG).In a nutshell,

%A CFG is a directed graph in which the vertices represent a linear sequence of program instructions and the arcs represent execution flow. Taking a global view from a program, it captures the program's control logic because all the possible execution paths are embedded in the graph. Machine learning algorithms can conveniently access such global characteristics via the CFG's adjacency matrix. On the other hand, local semantics of the program is also preserved via the instruction sequences inside the basic code blocks. This leaves us the ability to still selectively extract from vertices features heavily used and proven useful, for example n-grams\cite{NgramMalwareDetect} or q-grams\cite{QgramMalwareDetect}, in previous works. Extracting a vector of pre-defined attributes from each vertex quantifies a plain CFG as an Attributed Control Flow Graph (ACFG) without losing the structural information in CFG. The very first uniqueness of this work is to fully unitize the generic ACFG representation to classify malware families.

Classifying CFG-represented malware programs needs to address two types of performance issues,
classification performance ({\it does the malware classifier achieve high accuracy?}) and execution performance ({\it can the malware classifier work efficiently in practice?}).
As discussed earlier, reducing CFGs to vectors that contain simple aggregate features such as n-grams and opcodes leads to efficient malware classification but usually with poor classification accuracy.
Due to the graph nature of CFGs, previous works have studied use of graph similarity measures to train malware classification models~\cite{hu2009large,cesare2010classification,kong2013discriminant}.
However, some techniques for calculating pairwise graph similarities such as those based on graph matching or isomorphism can be computationally prohibitive,
letting alone that the time needed to compute pairwise graph similarity for a malware dataset scales quadratically with its size.

%and training malware classification models based on pairwise graph similarity data leads to quadratic scalability, making it inappropriate for large malware datasets.

%Before that, however, we still need the final piece of the puzzle: a graph-oriented algorithm capable of distinguishing graphs of different types.
%The graphs are of different number of vertices, generally not organized in order.
%The classification of graph is based on not only the attributes in all the vertices, but also the connection relationship between vertices.
%Previous works rely on pairwise graph similarity to make the call.
%For example, \cite{discovRe} adopted graph matching algorithms (for example, bipartite %graph matching) to tell if two graphs are similar or dissimilar.
%A later work by \cite{GraphEmbedding} start to utilize the embedding concept in deep learning to improve the efficiency of measuring the similarity of the graphs.
%Unfortunately, the scalability of pairwise similarity measurement is fundamentally limited by the quadratic time complexity as the number of graphs increases.
%Additionally, as the definition of similarity is highly depends on applications, works in this track are likely hard to adapt to varying scenarios.

Against this backdrop, in this work we propose to use a state-of-the-art deep learning infrastructure, graph kernel-based deep neural network, to classify malware programs represented as control flow graphs.
Due to their capability of understanding complex graph data, graph kernel-based deep neural network have found success in a number of application domains,
such as protein classification, chemical toxicology prediction, and cancer detection~\cite{ToxicologyPredict, ProteinDataset, SimonovskyEcc, Dgcnn}.
Particularly, our work extends a special type of graph kernel-based deep neural network, Deep Graph Convolutional Neural Network (DGCNN)~\cite{Dgcnn} for classifying CFG-represented malware programs.
Different from graph classification techniques based on pairwise graph similarity,
DGCNN allows attribute information associated with individual vertices to be aggregated quickly through neighborhood defined by the graph structure in breadth-first-search fashion,
thus embedding high-dimensional structural information into vectors that are amenable to efficient classification.

%apply the gradient descent optimization method directly on individual graph samples, which thus alleviates the necessity of embedding graphs into vectors explicitly before the classification step.

%We address the drawbacks of existing graph analysis approaches by introducing the %state-of-the-art deep learning architecture,
%graph kernel based deep neural network.
%It is a good fit for three reasons.
%First, as a set of neural network architectures that predict the class of a graphs on the basis of both graph structure and vertices (and/or edge) features,
%they already achieves fairly competitive performance in domains like protein classification, chemical toxicology prediction, movie collaboration and so on.
%This is attributed to graph kernel based neural network's generalization ability besides its capability to effectively understand the graphic data.
%More specifically we choose to extend the Deep Graph Convolutional Neural Network (DGCNN)\cite{Dgcnn}.
%Unlike the embedding approach\cite{GraphEmbedding} having to train on paired graphs, DGCNN allows end-to-end gradient descent on individual graph samples,
%without the need to first `embed' graph into vectors before classifying.
%Moreover, though our system builds on the existing DGCNN, we have taken a step forward by proposing several modifications to boost its effectiveness.

\if 0
Putting all together, in this paper we present \sysname,
an end-to-end \underline{M}alware detection system characterized by both \underline{A}ttributed control flow \underline{G}raph representation and
us\underline{I}ng deep graph \underline{C}onvolution neural network.
Designed as an data-oriented machine learning pipeline, we illustrate \sysname end-to-end in Figure~\ref{MG:Fig:SystemPipeline}.
Starting from the left end, \sysname assumes a raw executable binary is firstly disassembled into intermediate assembly code,
which is readily supported by many binary analysis tools, like IDA-Pro.
Cross-platform assembly code then go through a parser and a extractor that together builds the attributed control flow graph.
System flow ends with a malware prediction made by a pre-trained deep-learning based decision engine,
powered by our carefully re-engineered graph convolution neural network model.
We unveil the design details component by component from Section~\ref{MG:Sec:DGCNN} to Section~\ref{MG:Sec:Implement}.
\fi

To demonstrate the applicability of DGCNN for malware classification,
we have developed a new system called \sysname (an end-to-end \underline{ma}lware defense system that classifies CF\underline{G}-represented malware programs us\underline{i}ng DG\underline{C}NN).
\sysname improves the effectiveness of malware classification by extending the standard DGCNN with customized techniques tailored for malware classification.
We use two large malware datasets to evaluate the performance of \sysname and the experimental results show that it can classify CFG-represented malware programs with accuracy comparable to those of the state-of-the-art methods applied on handcrafted malware features.
The experimental results show that \sysname achieves performance comparable to state-of-the-art methods.

%The remainder of the paper is organized as follows.
%In Section~\ref{MG:Sec:System}, we introduce the overall design of \sysname.
%In Section~\ref{MG:Sec:DGCNN}, we provide a primer on DGCNN and then present our improvements over the standard DGCNN for classifying CFG-represented malware programs.
%In Section~\ref{MG:Sec:Implement} we discuss a few implementation details of \sysname.
%We present our experimental results in Section~\ref{MG:Sec:Experiment}.
%We draw concluding remarks in Section~\ref{MG:Sec:Conclusion}.

% In a summary, our contributions made in this work are as follows.
% \begin{itemize}
%     \item First we represent binary executable as the generic ACFG, which not only fully preserves the global semantics of the program,
%     but also extracts highly-customizable attributes of the instruction sequences inside local vertices.
%     \item We start from a state-of-the-art DGCNN to fuse the vertices features according to the graph structure. Our further extensions to the
%     existing neural network provides a certain degree of improvement comparing to the off-the-shelf model.
%     \item Combining the advantages of the ACFG representation and DGCNN, we propose \sysname, an end-to-end malware detection system,
%     and evaluate it with both public and private datasets of significant amount of malware samples.
%     Experiment result shows MalDenfenser achieves performance comparable to state-of-the-art methods.
% \end{itemize}

