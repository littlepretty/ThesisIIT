\Chapter{Conclusion}
\label{Sec:Conclusion}

Under the centre theme of securing SDN-enabled large-scale network using high-fidelity and scalable testing system, we conduct our research work in three streams.
First, we address one of the key issues in Linux-container-based network emulation (LCNE).
Even though LCNE combines many desired features of software simulation and physical testbeds,
it uses the system clock across all the containers even if a container is not being scheduled to run.
This leads to the issue of both performance and temporal fidelity, especially with high workloads.
Virtual time sheds the light on this issue by precisely scaling the time of interactions between containers and physical devices.
We develope a lightweight Linux-container-based virtual time system and integrate it to Mininet.
Except for enhancing Minint's fidelity and scalability,
our virtual time system is also an alternative for synchronizing clocks in hybrid simulation and emulation.
Second, we rethink how to simulate SDN network by taking advantage of the centralized paradigm.
Following this idea, we present a model abstraction technique that effectively transforms
the network devices in an SDN-based network to one virtualized switch model.
While significantly reducing the model execution time and enabling the real-time simulation capability,
our abstracted model also preserves the end-to-end forwarding behavior of the original network.
Third, motivated by the recent advancement and success of deep neural networks,
we study the feasibility of deep learning based network intrusion detection systems (NIDS) in order to enhance the essential network-architecture-agnostic security building-block.
We construct the detection engine with multiple advanced deep learning models and compare their performance.
