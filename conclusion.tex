
Under the central theme of securing SDN-enabled large-scale network using high-fidelity and scalable testing system, we conduct research works in three streams.

First, we address one of the key issues of Linux-container-based network emulation.
Even though it combines many desired features of software simulation and physical testbeds,
Linux container-based network emulation uses the system clock across all the containers even if a container is not being scheduled to run.
This leads to the issue of both performance and temporal fidelity, especially with high workloads.
Virtual time sheds the light on this issue by precisely scaling the time of interactions between containers and physical devices.
We developed a lightweight Linux-container-based virtual time system and integrate it to Mininet.
Except for enhancing Mininet's fidelity and scalability,
our virtual time system also plays a key role for synchronizing clocks in hybrid simulation and emulation testbed.

Second, we rethink how to simulate SDN network by taking advantage of the centralized paradigm.
Following this idea, we present a model abstraction technique that effectively transforms
the network devices in an SDN-based network to one virtualized switch model.
While significantly reducing the model execution time and enabling the real-time simulation capability,
our abstracted model also preserves the end-to-end forwarding behavior of the original network.

Third, motivated by the recent advancement and success of deep neural networks,
we study the feasibility of deep learning technologies for enhancing the essential network-architecture-agnostic security building-block.
We construct the detection engine with multiple advanced deep learning models and compare their performance.
We take a step further from the comparative study.
Our study shows that existing machine learning-based host instruction detection methods rely on handcrafted features extracted from raw binary files or disassembled code.
The diversity of such features created has made it hard to build generic malware classification systems that
work effectively across different operational environments.
To strike a balance between generality and performance,
we explore new deep learning techniques to effectively yet efficiently classify malware programs represented as their control flow graphs.
The resultant innovative system uses deep graph convolutional neural network to
embed structural information inherent in control flow graphs for effective yet efficient malware classification.

Following each explored research direction respectively, we envision the following possible future works.
First, for the virtual time system, we may find some use case that motivate us to integrate container-based emulation into an optimistic parallel discrete event simulation.
For example, many high performance computing (HPC) researchers adopt optimistic parallel discrete event simulation~\cite{CodesSim}
to evaluate the performance of HPC system's network layer, and how application layer is affected by network layer.
When evaluating a HPC application's performance on dragonfly network,
comparing to the current methodology that runs application as a sequence of discrete events extracted from traces collected when executed previously on possible different network topology,
directly executing the application running on top of simulated target network topology may obtain more accurate results,
and may even reveal interactive facts between application layer and network layer that pure simulation fails to capture.
However, there is an obvious technical challenge one needs to address in coordination of rollbackable simulation and emulation:
when a rollback happens in optimistic parallel discrete event simulation, in order to maintain the correctness of the evaluation,
the coordinator may require the emulation to rollback as well.
The rollback operation for emulation includes both restore state variables and clock.
Virtual time system could help the coordinator of simulation and emulation easily adjust the time of the emulated containers.

The ``one-big-switch'' model abstraction only ensures the end-to-end forwarding logic equivalence.
This is not sufficient for many SDN network designers or operators, who want to run high fidelity simulation to measure target network's performance,
including but not limited to, end-to-end throughput, end-to-end latency, and end-to-end packet loss rate.
It is a natural extension to consider these user-interested network performance metrics by introducing end-to-end performance equivalence.

The intrusion detection systems proposed in this thesis are targeting on network traffics from general TCP/IP network environment,
and executable binaries running on standard operating systems.
While our approaches are still able to protect the SDN-enabled smart grid in both communication layer and the application layer,
they may not achieve required performance due to over-generalization.
There are unique challenges, most likely related to issues from data,
we need to identify, and then address in order to improve underfitted machine learning models.

