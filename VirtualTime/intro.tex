\section{Virtual Time Support for Network Emulation}
\label{VT:Sec:Intro}

Linux-container-based network emulation (LCNE) combines many desired features of software simulators and physical testbeds.
Therefore, first we ask not what LCNE can do for computer network -- ask what we can do for LCNE.
Specifically, we address one of the key issues of this attractive methodology.
An ordinary Linux-container based SDN emulator uses the system clock across all the containers even if a container is not being scheduled to run.
This leads to the issue of both performance and temporal fidelity, especially when handling high workloads during emulation.

Our approach is to develop the notion of virtual time inside containers to improve fidelity and scalability of the container-based network emulation.
A key insight is to trade time for system resources by precisely scaling the system's capacity to match behaviors of the target network.
The idea of virtual time has been explored in the form of time-dilation-based~\cite{ToInfinityBeyond} and
VM-scheduling-based~\cite{VirtTimeOpenVZ, SliceTime} designs, and has been applied to various virtualization platforms including Xen~\cite{DieCast},
OpenVZ~\cite{VirtTimeOpenVZ}, and Linux Container~\cite{TimeKeeper}.
In this work, we take a time-dilation-based approach to build a lightweight virtual time system in Linux container,
and have integrated the system to Mininet for scalability and fidelity enhancement.
The time dilation factor (TDF) is defined as the ratio between the rate at which wall-clock time has
passed to the emulated host's perception of time~\cite{ToInfinityBeyond}.
A TDF of 10 means that for every ten seconds of real time, applications running in a time-dilated emulated host perceive the time advancement as one second.
This way, a 100 Mbps link is scaled to a 1 Gbps link from the emulated host's viewpoint.
Our system transparently provides the virtual time to processes inside the containers,
while returns the ordinary system time to other processes.
No change is required in applications, and the integration with network emulators is easy (only slight changes in the initialization routine).
Benchmark evaluation and data center case study demonstarte that, except for enhanceing Mininet's fidelity and scalability,
our virtual time system is also an good alternative for the synchronization problem in hybrid simulation and emulation.

