\section{Virtual Time System}
\label{VT:Sec:RelatedWorks}

Virtual time has been investigated to improve the scalability and fidelity of virtual-machine-based network emulation. 
There are two main approaches to develop virtual time systems.
The first approach is based on time dilation, a technique to uniformly scale the virtual machine's perception of time by a specified factor. 
It was first introduced by Gupta et al.~\cite{ToInfinityBeyond},
and has been adopted to various types of virtualization techniques and integrated with a handful of network emulators. 
Examples include DieCast~\cite{DieCast}, SVEET~\cite{SVEET}, NETbalance~\cite{NETbalance}, TimeJails~\cite{ComparisonVR-VM, TimeJails} and TimeKeeper~\cite{TimeKeeper}. 
The second approach focuses on synchronized virtual time by modifying the hypervisor scheduling mechanism. 
Hybrid testing systems that integrate network emulation and simulation have adopted this approach. 
For example,~\cite{jin2012virtual} integrates an OpenVZ-based virtual time system~\cite{VirtTimeOpenVZ} with
a parallel discrete-event network simulator by virtual timer.
SliceTime~\cite{SliceTime} integrates ns-3~\cite{NS-3} with Xen to build a scalable and accurate network testbed.

Our approach is technically closest to TimeKeeper~\cite{TimeKeeper} through direct kernel modifications of time-related system calls. 
The differences are (1) we are the first to apply virtual time in the context of SDN emulation,
(2) our system has a wider coverage of system calls interacting in virtual time,
and (3) our system has an adaptive time dilation scheduler to balance speed and fidelity for emulation experiments.

\subsection{Adaptive Virtual Time Scheduling}
The key insight of virtual time is to trade time with system resources. 
Therefore, a primary drawback is the proportionally increased execution time.
To determine an appropriate TDF, VENICE~\cite{VirtualTimeMachine} proposes a static management scheme to forecast the recourse demand. 
One problem of static time dilation management is that we often assume the maximum load to ensure fidelity and thus overestimate the scaling factor. 

TimeJails \cite{TimeJails} presents a dynamic management scheme~\cite{NtwkEmultAdaptVirtTime} to adjust TDF in run-time based on CPU utilization. 
We take a similar approach with two differences: the target platform and communication overhead.
TimeJails is a Xen-based platform extended to a 64-node cluster for scalability,
while our system supports more scalable experiments on a single machine with Linux container.
TimeJails requires a special protocol to prioritizing TDF request message in local area networks,
while the communication overhead of our system is much lower, either through synchronized queues or method invocations among extended modules in the emulator.

\subsection{SDN Emulation and Simulation}
OpenFlow~\cite{Openflow} is the first standard communications interface defined between the control and forwarding planes of an SDN architecture. 
Examples of OpenFlow-based SDN emulation testbeds include Mininet~\cite{LaptopSDN},
Mininet-HiFi~\cite{ReproNetExprCBE}, EstiNet~\cite{EstiNet}, ns-3~\cite{NS-3}, fs-sdn~\cite{FSSDN}, OpenNet~\cite{OpenNet}, and S3FNet~\cite{jin2013parallel}.
Mininet is currently the most popular SDN emulator, which uses Linux-container-based virtualization technique to provide a lightweight, high-fidelity and inexpensive testbed.
NS-3~\cite{NS-3} has an OpenFlow simulation model and also offers a realistic OpenFlow environment through its generic emulation capability.
It also offers simulation models of SDN networks and emulation of SDN controllers via the direct code execution (DCE) technique.
S3FNet~\cite{jin2013parallel} supports scalable simulation/emulation of OpenFlow-based SDN through
a hybrid platform that integrates a parallel network simulator and a virtual-time-enabled OpenVZ-based network emulator~\cite{S3FWebsite}.
fs-sdn~\cite{FSSDN} extends fs, a flow-level discrete event network simulator, with the SDN capability.
