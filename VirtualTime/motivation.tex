\section{Motivation}
\label{VT:Sec:Intro}

Researchers conducting analysis of networked computer systems are often concerned with questions of scale.
What is the impact of a system if the communication delay is X times longer, the bandwidth is Y times larger, or the processing speed is Z times faster? 
Various testbeds have been created to explore answers to those questions before the actual deployment.
Ideally, testing on an exact copy of the original system preserves the highest fidelity,
but is often technically challenging and economically infeasible, especially for large-scale systems. 
Simulation can significantly improve the scalability and reduce the cost by modeling the real systems.
However, the fidelity of modeled systems is always in question due to model abstraction and simplification. 
For example, large ISPs today prefer to evaluate the influence of planned changes of their internal networks
through tests driven by realistic traffic traces rather than complex simulations. 
Network emulation is extremely useful for such scenarios, by allowing unmodified network
applications being executed inside virtual machines (VMs) over controlled networking environments.
This way, scalability and fidelity is well balanced as compared with physical or simulation testbeds.

A handful of network emulators have been created based on various types of virtualization technologies. 
Examples include DieCast~\cite{DieCast}, TimeJails~\cite{TimeJails}, VENICE~\cite{VirtualTimeMachine} and dONE~\cite{RelativisticTime},
built upon full or para-virtualization (such as Xen~\cite{Xen}), as well as Mininet~\cite{LaptopSDN, ReproNetExprCBE}, CORE~\cite{CORE} and vEmulab~\cite{Emulab},
using OS-level virtualization (such as OpenVZ~\cite{OpenVZ}, Linux container~\cite{LXC} and FreeBSD jails~\cite{FreeBSDJails}).
All those network emulators offer functional fidelity through the direct execution of unmodified code. 
Xen enables virtualization of different operating systems, whereas lightweight Linux container enables virtualization at the application level with two orders of magnitude more of VM (or container) instances, i.e., emulated nodes, on a single physical host, in the cost of only able to run a single type of operating system. 
In this work, we focus on improving the Linux container technology for scalable network emulation, in particular with the application of software-defined networks (SDN). 
Mininet~\cite{LaptopSDN} is by far the most popular network emulator used by the SDN community~\cite{Frenetic, AbsNetUpd, LivMigEntNet}. 
The Linux-container-based design enables Mininet users to experiment ``a network in a laptop" with thousands of emulated nodes. 
However, Mininet cannot guarantee fidelity at high loads, in particular when the number of concurrent active events is more than the number of parallel cores. 
For example, on a commodity machine with 2.98 GHz CPU and 4 GB RAM providing 3 Gb/s internal bandwidth,
Mininet is only capable to emulate a network up to 30 hosts, each with a 100 MHz CPU and 100 MB RAM and connected by 100 Mb/s links~\cite{ReproNetExprCBE}. 
Emulators cannot reproduce correct behaviors of a real network with large topology and high traffic load because of the limited physical resources. 
In fact, the same issue occurs in many other VM-based network emulators, because a host \emph{serializes} the execution of multiple VMs, rather than in parallel like a physical testbed. 
VMs take its notion of time from the host system's clock, and hence time-stamped events generated by the VMs are multiplexed to reflect the host's serialization.

Our solution is to develop the notion of virtual time inside Linux containers.
The key insight is to trade time for system resources by precisely scaling the system's capacity to match behaviors of the target network.
Specifically, our contributions to this research problem are summarized as follows. 
First, we have developed an independent and lightweight middleware in the Linux kernel to support virtual time for Linux container. 
Our system transparently provides the virtual time to processes inside the containers, while returns the ordinary system time to other processes. 
No change is required in applications, and the integration with network emulators is easy (only slight changes in the initialization routine). 
Second, to the best of our knowledge, we are the first to apply virtual time in the context of SDN emulation, and have built a prototype system in Mininet. 
Experimental results indicate that with virtual time, Mininet is capable to precisely emulate much larger networks with high loads, approximately increased by a factor of TDF. 
Third, we have designed an adaptive time dilation scheme to optimize the performance tradeoff between speed and fidelity. 
Finally, we have demonstrated the fidelity improvement through a realistic case study about evaluation of the limitations of
the equal-cost multi-path (ECMP) routing protocol in data center networks.

The remainder of the chapter is structured as follows. Section~\ref{VT:Sec:Architecture} presents the virtual time system architecture design. 
Section~\ref{VT:Sec:Implementation} illustrates the implementation of the system and its integration with Mininet.
Section~\ref{VT:Sec:Experiments} evaluates the virtual-time-enabled Mininet, with a case study of ECMP routing evaluation in Section~\ref{VT:Sec:CaseStudy}. 
Section~\ref{VT:Sec:Conclusion} summarize this chapter.
