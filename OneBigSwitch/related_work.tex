%\section{Related Work}
%\label{OBS:Sec:RelatedWork}

\subsection{SDN Forwarding Rules Abstraction}

The idea of ``one big switch" is originated from~\mbox{\cite{OneBigSwitchAbstraction}} for a different purpose.
In their work, the one-big-switch network abstraction is used to reduce conflicting rules generated by
various high-level SDN applications that simultaneously run on one or even multiple controllers.
Their system takes an optimization-based approach to solve the rule placement problem with
the objective of minimizing the number of rules that need to be installed in forwarding devices.
Application developers are now shielded from the rules distributed across switches, and only need to specify the end-to-end policies on the big switch model. 
The objective of our work on the other hand is to reduce the model execution time and to
enhance the scalability of network simulation and emulation.
We take a different technical approach based on statically analyzing snapshots of the network state to generate rules in the big switch abstraction model.
There exists a line of research on network fault detection by analyzing software,
configuration and network-wide data-plane state~\cite{Al-Shaer2010,Al-Shaer2009,Anteater2011,xz+05}.
Those approaches typically operate offline on timescales of seconds to hours.
Real time network verification tools are developed to enforce correctness in connectivity \cite{NetPlumber2013,Veriflow}.
\if 0
\hl{
NetPlumber~\mbox{\cite{NetPlumber2013}} uses a ``header space analysis" model to
describe forwarding behaviors as transformation on arbitrary header bits.
It has very elegant geometric interpretation but, unlike Veriflow~\mbox{\cite{Veriflow}}, did not
provide data structures to aggregate or organize the classes of packets.
}
\fi
Our work leverages the idea of slicing the entire network into equivalence classes in~\cite{Veriflow}
to reduce the problem space, which enables fast model abstraction execution speed.
%We consider the entire connected SDN network when we run forwarding traversal, especially the boundary switches that Veriflow neglects. We also add up several algorithms in finding equivalence classes.

%\subsection{SDN Emulation and Simulation}
%There are a number of SDN emulation and simulation testbeds based on the OpenFlow
%protocol.
%Examples include Mininet~\cite{Mininet}, EstiNet~\cite{EstiNet}, ns-3~\cite{NS3},
%S3FNet~\cite{S3F_website}, fs-sdn~\cite{FSSDN} and OpenNet~\cite{OpenNet}.
%Mininet  \cite{Mininet} applies container-based virtualization technique and cgroup based resource isolation to provide a lightweight and high fidelity emulation
%platform.
%Its functional fidelity is guaranteed by executing real SDN switch/controller software.
%ns-3~\cite{NS3} offers simulation models of SDN networks and emulation of SDN controllers via the direct code execution (DCE) technique.
%S3FNet~\cite{S3F_website} is a hybrid OpenFlow-based SDN testing platform that integrates a parallel network simulator with an OpenVZ-based network emulator.
%fs-sdn~\cite{FSSDN} extends fs, a flow-level discrete event network simulator, with the SDN capability.
%We develop a model abstraction method in this paper to transform a large scale and complicated SDN network to a one-big-switch-based network.
%We can use the resulting abstracted network model in all the aforementioned simulation and emulation environment for performance gain while still preserving the network forwarding logic.
