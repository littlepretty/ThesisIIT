\section{Motivating Example}
\label{OBS:Sec:MotivatingExample}

\tikzset {
    roundnode/.style={circle, draw=blue!80, fill=blue!5, very thick, minimum size=7mm},
        squarenode/.style={rectangle, draw=red!80, fill=red!5, very thick, minimum size=7mm},
        ->-/.style={->, line width=1.2pt},
        ==/.style={line width=1.6pt}
    }

\begin{figure}[t]
    \centering
    \begin{tikzpicture}
        \node[roundnode] (sw0) {SW0};
        \node[roundnode] (sw1) [below left=2cm of sw0] {SW1};
        \node[roundnode] (sw2) [below=1.1cm of sw0] {SW2};
        \node[roundnode] (sw3) [below right=2cm of sw0] {SW3};
        \node[squarenode] (net1) [below=1cm of sw1] {10.1.0.10/16};
        \node[squarenode] (net2) [below=1cm of sw2] {10.0.0.10/8};
        \node[squarenode] (net3) [below=1cm of sw3] {11.1.0.10/16};

        \draw[-, line width=1.2pt] (sw0) -- node [near start, above] {0} node[near end, below] {1} (sw1);
        \draw[-, line width=1.2pt] (sw0) -- node [near start, right] {1} node[near end, left] {1} (sw2);
        \draw[-, line width=1.2pt] (sw0) -- node [near start, above] {2} node[near end, below] {1} (sw3);
        \draw[-, line width=1.2pt] (sw1) -- node [near start, left] {0} (net1);
        \draw[-, line width=1.2pt] (sw2) -- node [near start, left] {0} (net2);
        \draw[-, line width=1.2pt] (sw3) -- node [near start, right] {0} (net3);
    \end{tikzpicture}
    \caption{A Tree-Topology SDN Network}
    \label{OBS:Fig:ExampleNetworkTopo}
\end{figure}

\begin{figure*}
    \centering
    \subfigure[Forwarding Graph for EC1 and EC3] {
        \begin{tikzpicture}
            \node[roundnode] (sw2) {SW2};
            \node[roundnode] (sw0) [left=of sw2] {SW0};
            \node[roundnode] (sw1) [above left=of sw0] {SW1};
            \node[roundnode] (sw3) [below left=of sw0] {SW3};
            \node[squarenode, align=center] (ec13) [right=of sw2] {EC1: 10.0.*.*\\EC3: 10.2.0.0-10.255.255.255};

            \draw[->-] (sw0) -- node [near start, above] {1} node[near end, above] {1} (sw2);
            \draw[->-] (sw1) -- node [near start, above] {1} node[near end, above] {0} (sw0);
            \draw[->-] (sw3) -- node [near start, above] {1} node[near end, above] {2} (sw0);
            \draw[->-] (sw2) -- node [near start, above] {0} (ec13);
        \end{tikzpicture}
    \label{OBS:Fig:ExampleForwardingEC13}
}

    \subfigure[Forwarding Graph for EC2] {
        \begin{tikzpicture}
            \node[roundnode] (sw1) {SW1};
            \node[roundnode] (sw0) [left=of sw2] {SW0};
            \node[roundnode] (sw2) [above left=of sw0] {SW2};
            \node[roundnode] (sw3) [below left=of sw0] {SW3};
            \node[squarenode] (ec2) [right=of sw1] {EC2: 10.1.*.*};

            \draw[->-] (sw0) -- node [near start, above] {0} node[near end, above] {1} (sw1);
            \draw[->-] (sw2) -- node [near start, above] {1} node[near end, above] {1} (sw0);
            \draw[->-] (sw3) -- node [near start, above] {1} node[near end, above] {2} (sw0);
            \draw[->-] (sw1) -- node [near start, above] {0} (ec2);
        \end{tikzpicture}
    \label{OBS:Fig:ExampleForwardingEC2}
}
    \subfigure[Forwarding Graph for EC4] {
        \begin{tikzpicture}
            \node[roundnode] (sw3) {SW3};
            \node[roundnode] (sw0) [left=of sw3] {SW0};
            \node[roundnode] (sw1) [above left=of sw0] {SW1};
            \node[roundnode] (sw2) [below left=of sw0] {SW2};
            \node[squarenode] (ec4) [right=of sw3] {EC4: 11.1.*.*};

            \draw[->-] (sw0) -- node [near start, above] {2} node[near end, above] {2} (sw3);
            \draw[->-] (sw1) -- node [near start, above] {1} node[near end, above] {0} (sw0);
            \draw[->-] (sw2) -- node [near start, above] {1} node[near end, above] {1} (sw0);
            \draw[->-] (sw3) -- node [near start, above] {0} (ec4);
        \end{tikzpicture}
    \label{OBS:Fig:ExampleForwardingEC4}
}
    \caption{Forward Graph for Each EC}
    \label{OBS:Fig:ExampleForwardingGraphs}
\end{figure*}

\begin{table*}[t]
    \caption{Forwarding Rules in the 3-ary Tree Network}
    \begin{center}
        \begin{tabular}{c|clc}
            \hline
            Switch & Priority & Match Field & Action\\
            \hline
            \hline
            \multirow{3}{2em}{SW0} & 10 & NW\_DST=10.1.*.* & FWD: OUT\_PORT=0 \\
                       & 1  & NW\_DST=10.*.*.* & FWD: OUT\_PORT=1 \\
                       & 1  & NW\_DST=11.1.*.* & FWD: OUT\_PORT=2 \\
            \hline
            \multirow{3}{2em}{SW1} & 10 & IN\_PORT=1, NW\_DST=10.1.*.* & FWD: OUT\_PORT=0 \\
                       & 1  & IN\_PORT=0, NW\_DST=10.*.*.* & FWD: OUT\_PORT=1 \\
                       & 1  & IN\_PORT=0, NW\_DST=11.1.*.* & FWD: OUT\_PORT=1 \\
            \hline
            \multirow{3}{2em}{SW2} & 10 & IN\_PORT=0, NW\_DST=10.1.*.* & FWD: OUT\_PORT=1 \\
                       & 1  & IN\_PORT=1, NW\_DST=10.*.*.* & FWD: OUT\_PORT=0 \\
                       & 1  & IN\_PORT=0, NW\_DST=11.1.*.* & FWD: OUT\_PORT=1 \\
            \hline
            \multirow{3}{2em}{SW3} & 10 & IN\_PORT=1, NW\_DST=11.1.*.* & FWD: OUT\_PORT=0 \\
                       & 1  & IN\_PORT=0, NW\_DST=10.*.*.* & FWD: OUT\_PORT=1 \\
                       & 1  & IN\_PORT=0, NW\_DST=10.1.*.* & FWD: OUT\_PORT=1 \\
            \hline
        \end{tabular}
    \end{center}
    \label{OBS:Tab:OriginalFlowTable}
\end{table*}

\begin{table*}[t]
    \caption{Forwarding Rules on the ``Big OpenFlow Switch"}
    \begin{center}
        \begin{tabular}{c|clc}
            \hline
            Switch & Priority & Match Field & Action\\
            \hline
            \hline
            \multirow{4}{2em}{SW}  & 10 & NW\_DST=10.0.*.* & FWD OUT\_PORT=1 \\
                       & 10 & NW\_DST=10.1.*.* & FWD OUT\_PORT=0 \\
                       & 10 & NW\_DST=11.1.*.* & FWD OUT\_PORT=2 \\
                       & 10 & NW\_DST=10.2.0.0-10.255.255.255 & FWD OUT\_PORT=1 \\
            \hline
        \end{tabular}
    \end{center}
    \label{OBS:Tab:CompressedFlowTable}
\end{table*}

In this section, we describe our model abstraction technique to transform an SDN network model to a big-switch model with a concrete network example.
Let us consider a tree-topology network connected by four OpenFlow switches, as shown in Figure~\ref{OBS:Fig:ExampleNetworkTopo}.
The centralized SDN controller (not shown in the figure for simplicity) installs the forwarding rules on each switch to establish connections for all three subnets.
All the switch rules are shown in Table~\ref{OBS:Tab:OriginalFlowTable}.
We assume that during the process of model abstraction, the rules have been installed on each OpenFlow switch,
and there is no link down or rule modification. OpenFlow switch 0 (i.e., SW0) works as an \textbf{aggregation switch}
that provides connectivity for other switches. SW1, SW2 and SW3 work as \textbf{edge switches} that provide connectivity for each subnet,
and each edge switch connects to one end-host.

Our approach abstracts the network to one big switch that has logically equivalent forwarding behavior.

The first step in the abstraction process is to extract \textit{equivalence classes} through the OpenFlow rules installed on network devices, i.e., aggregation and edge switches.
%As formally defined later in Section~\ref{Sec:Design} and in \cite{Veriflow},
Equivalence class (EC) is the set of packets that experience identical forwarding action at \textbf{all} network devices.
%This concept is proposed to confine network verification activities to the minimal effected set of packets\cite{Veriflow}.
We utilize EC to merge all the rules on a set of switches. For example, the flow rules shown in Table~\ref{OBS:Tab:OriginalFlowTable} can be sliced into
four \textbf{disjoint} ECs based on the NW\_DST field as follows. Note that the matching field IN\_PORT cannot be used in identifying ECs,
since it is not a packet-dependent, but topology-dependent field.

\begin{itemize}
    \item Packets in EC1 are destined to the network address 10.0.*.*.
    \item Packets in EC2 are destined to hosts with address 10.1.*.*.
    \item Packets in EC3 are destined to the address range from 10.2.0.0 to 10.255.255.255
    \item Packets in EC4 are destined to the subnet 11.1.*.*.
\end{itemize}

After identifying all the ECs from the rule set, we generate \textit{forwarding graph} for each EC, which models how packets within an EC are forwarded through the network~\cite{Veriflow}.
The node in a forwarding graph represents a network device, and the directed edge represents how the network device forwards the packets.
The sink nodes, i.e., the red rectangle nodes in Figure~\ref{OBS:Fig:ExampleForwardingGraphs}, indentifies the EC that this forwarding graph belongs to.
Each equivalence class will have exactly one forwarding graph, as shown in Figure~\ref{OBS:Fig:ExampleForwardingGraphs}.
%If we skip the process of minimizing number of ECs from the rules,
Note that \{EC1, EC2, EC3, EC4\} is not yet the minimal set of ECs in the network.
In fact, EC1 and EC3 can be merged because the forwarding behaviors of both ECs are identical at any device in the network,
as depicted in Figure~\ref{OBS:Fig:ExampleForwardingEC13} that EC1 and EC3 share the same forwarding graph.

We finish the model abstraction by generating a new set of forwarding rules that are to be installed on the big switch.
To make the process more efficient, we only have to consider those ECs whose packets traverse edge switches in the network.

Table~\ref{OBS:Tab:CompressedFlowTable} shows the resulting rules that will be installed in the big switch (see Figure~\ref{OBS:Fig:ExampleBigSwitch}).
The resulting one-big-switch network has the identical forwarding functions to the original tree network from the end-to-end communication perspective.
% in the eyes of the three hosts(possibly emulated or simulated).
The number of switches we need to simulate or emulate is now reduced from four to one, and the number of rules in the network is reduced from twelve to four.
If we only consider OpenFlow rules that match the NW\_DST field and the action is always forwarding,
then the total number of rules in the big switch is proportional to the number of ECs,
whereas in the original SDN network, the total number of rules is $O(S\times P)$,
where $S$ is the number of switches and $P$ is the number of address prefixes.

\begin{figure}[t]
    \centering
    \begin{tikzpicture}
        \node[roundnode, minimum size=15mm, align=center] (sw) {Big\\Switch};
        \node[squarenode, align=center] (ec13) [below=1.3cm of sw] {10.0.0.10/8};
        \node[squarenode] (ec2) [below left=2cm of sw] {10.1.0.10/16};
        \node[squarenode] (ec4) [below right=2cm of sw] {11.1.0.10/16};

        \draw[-, line width=1.2pt] (sw) -- node [near start, above] {0} (ec2);
        \draw[-, line width=1.2pt] (sw) -- node [near start, right] {1} (ec13);
        \draw[-, line width=1.2pt] (sw) -- node [near start, above] {2} (ec4);
    \end{tikzpicture}
    \caption{Compressed SDN Network for Scalable Simulation}
    \label{OBS:Fig:ExampleBigSwitch}
\end{figure}
