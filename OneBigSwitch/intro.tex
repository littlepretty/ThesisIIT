\Section{One-Big-Switch Abstraction}
\label{OBS:Sec:Intro}

Second, we ask not what simulation will do for SDN-enabled network, but ask what the centralized paradigm can do for the methodology that simulates itself.
Following this idea, we present a model abstraction technique that effectively transforms
a SDN-based network model to a ``one-big-switch'' network model.
The idea was inspired by the work on rule placement optimization in~\cite{OneBigSwitchAbstraction}.
With the highly abstracted network, SDN application developers now only need to consider
simple end-to-end policy when programming a network,
and are shielded from the details on routing policy, switch memory limits,
and distributing rules across switches.

Our work applies the idea of one-big-switch abstraction for enhancing the scalability
of network simulation and emulation, while preserving the end-to-end forwarding behaviors of the original network.
This technique is useful if users only care about the end-to-end behavior rather than
the details within the network, such as hop-by-hop routing, or table lookup on each single switch.
For example, users may want to simulate a large-scale complex network of networks consisted of multiple types of network protocols.
While significantly reducing the model execution time, the one-big-switch model also enables the real-time simulation capability,
which execute the network models no slower than the wall-clock time.
In addition, the abstracted model can be shared by different parties that do not want to disclose the details
of their own production network.

We develop a three-step approach to transform an SDN network to a big OpenFlow switch based network,
while still preserving the network forwarding logic equivalence.
% The high-level idea is illustrated in Figure~\ref{OBS:Fig:BigSimOverview},
% and the details are discussed in Section~\ref{OBS:Sec:Design}.
We first group all packets into equivalence classes by analyzing the matching fields
(e.g., source/destination MAC address/IP address/port, VLAN id, etc.)
of the OpenFlow rules installed on the switches.
An equivalence class represents a set of packets of the same network forwarding behavior.
We then create a graph-based model for each equivalence class to model its packet forwarding behavior.
Finally, we traverse all the forwarding graph models to generate rules for the big switch,
and the number of rules is largely reduced but original network's end-to-end forwarding behavior is not affected.

